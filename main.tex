\documentclass[12pt]{ctexart}
\usepackage{graphicx} % Required for inserting images
\usepackage{subcaption}

\begin{document}

由于随机种子 112401520,112401521,……,112401525 均无解,选择 112401526 作为随机种子。

图 \ref{fig: main_1} 展示了运行结果,其中图 \ref{fig: p1} 是初始地图,图 \ref{fig: p2} 是 A* 算法的结果,图 \ref{fig: p3} 是 BFS 算法的结果。

图 \ref{fig: main_2} 展示了可视化结果,其中图 \ref{fig: p4} 是 A* 算法的可视化,图 \ref{fig: p5} 是 BFS 算法的可视化。

从下列图中可知,A* 和 BFS 算法都能寻找到迷宫的最短路径,但 A* 算法需要遍历的迷宫格子数量更少,而 BFS 算法需要遍历的迷宫格子数量更多,体现了 A* 算法的效率高于 BFS 算法。

\begin{figure}[htp]
    \centering
    \begin{subfigure}{0.8\textwidth}
        \centering
        \includegraphics[width = \textwidth, keepaspectratio]{pic1.png}
        \caption{学号和初始地图}
        \label{fig: p1}
    \end{subfigure}
    \begin{subfigure}{0.8\textwidth}
        \centering
        \includegraphics[width = \linewidth, keepaspectratio]{pic2.png}
        \caption{A* 算法结果}
        \label{fig: p2}
    \end{subfigure}
    \hfill
    \begin{subfigure}{0.8\textwidth}
        \centering
        \includegraphics[width = \linewidth, keepaspectratio]{pic3.png}
        \caption{BFS 算法结果}
        \label{fig: p3}
    \end{subfigure}
    \caption{运行结果}
    \label{fig: main_1}
\end{figure}

\FloatBarrier

\begin{figure}[htp]
    \centering
    \begin{subfigure}{0.8\textwidth}
        \centering
        \includegraphics[width = \textwidth, keepaspectratio]{figure_astar.png}
        \caption{A* 算法可视化结果}
        \label{fig: p4}
    \end{subfigure}
    \hfill
    \begin{subfigure}{0.8\textwidth}
        \centering
        \includegraphics[width = \textwidth, keepaspectratio]{figure_bfs.png}
        \caption{BFS 算法可视化结果}
        \label{fig: p5}
    \end{subfigure}
    \caption{matplotlib 可视化}
    \label{fig: main_2}
\end{figure}

\end{document}
